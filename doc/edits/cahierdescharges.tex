\documentclass[11pt, a4paper]{article}

\usepackage[francais]{babel}
\usepackage[T1]{fontenc}
\usepackage[utf8]{inputenc}

\usepackage[left=2cm, right=2cm, top=2cm, bottom=2cm]{geometry}
\usepackage{fancyhdr}
\usepackage{lastpage}
\usepackage{hyperref}

\usepackage{graphicx}
\usepackage{tikz}
\usepackage{pgfgantt}

\usepackage{color}
\usepackage{colortbl}

\usepackage{makeidx}

\usepackage{listings}
\usepackage{CJKutf8}

\makeindex
\makeatletter
\renewenvironment{theindex}
{
    \thispagestyle{fancy}
    \setlength{\parindent}{0pt}
    \let
    \item
    \@idxitem
}{\onecolumn}
\makeatother

\setlength{\parindent}{0cm}

\lstset{
    basicstyle=\ttfamily,
    stringstyle=\ttfamily\color{green!50!black},
    keywordstyle=\bfseries\color{blue},
    commentstyle=\itshape\color{red!50!black},
    showstringspaces=true,
    tabsize=4,
    frame=single,
    numbers=left,
    numberstyle=\tiny,
    firstnumber=1,
    stepnumber=1,
    numbersep=5pt,
    breaklines=true
}

\graphicspath{{img/}}

\pagestyle{fancy}
\setlength{\headheight}{14pt}
\renewcommand{\headrulewidth}{0.5pt}
\lhead{Yannick Brodard}
\chead{}
\rhead{\today}
\renewcommand{\footrulewidth}{0.5pt}
\lfoot{CFPT - École d'informatique}
\cfoot{Cahier des charges}
\rfoot{Page \thepage ~sur \pageref{LastPage}}

% Niveau table des matières
\setcounter{tocdepth}{3}

% ---------------------------------------------------------------------

% Titre
\title{
\Huge{Étude de Unreal Engine 4} \\[0.5cm]
\LARGE{Cahier des charges}
}
\author{Yannick Brodard}
\date{\today}

\begin{document}
\maketitle
\begin{center}
    \includegraphics[scale=0.4]{UE4_logo}
\end{center}
\thispagestyle{empty}
\newpage
\section{Objectifs du projet}
Les objectifs de ce travail sont :
\begin{itemize}
\item Prendre en main l'environnement Unreal Engine 4
\item Mettre en avant les fonctionnalités intéressantes de cet environnement
\item Réaliser un mini-jeu \textit{Galaga}-like
\item Produire une documentation technique du projet
\item Maintenir un journal de bords pendant la réalisation du projet
\end{itemize}
\section{Description détaillée}
Le but est d'étudier l'environnement de Unreal Engine 4 et de produire un mini-jeu de genre \textit{Galaga}. Il faudra connaître cet environnement et mettre en avant ses fonctionnalités.

Le mini-jeu est basé sur le jeu \textit{Galaga}. Dans ce jeu 2D, l'utilisateur contrôle un vaisseau qui peut se déplacer sur les axes x et y. Des vagues de vaisseaux ennemis vous attaquent jusqu'à ce que l'utilisateur les détruise tous.
\section{Inventaire du matériel}
\begin{itemize}
\item PC 1x
	\begin{itemize}
	\item 1024MB NVIDIA GeForce GTX 550 Ti
	\item Intel Core i7 2600K @ 3.40GHz
	\item ASUSTeK Computer INC. P8Z68-V LE
	\item 8.00 Go Dual-Channel DDR3 @ 824MHz
	\end{itemize}
\item Écran 2x
\item Clavier 1x
\item Souris 1x
\end{itemize}
\section{Inventaire des logiciels}
\begin{itemize}
\item Unreal Engine 4
\end{itemize}
\section{Éléments mesurables (servant à l'évaluation)}
Réalisation des objectifs, mesurés selon la grille d'évaluation
\end{document}
