\documentclass[11pt, a4paper]{article}

\usepackage[francais]{babel}
\usepackage[T1]{fontenc}
\usepackage[utf8]{inputenc}

\usepackage[left=2cm, right=2cm, top=2cm, bottom=2cm]{geometry}
\usepackage{fancyhdr}
\usepackage{lastpage}
\usepackage{hyperref}

\usepackage{graphicx}
\usepackage{tikz}
\usepackage{pgfgantt}

\usepackage{color}
\usepackage{colortbl}

\usepackage{makeidx}

\usepackage{listings}
\usepackage{CJKutf8}

\usepackage{float}

\makeindex
\makeatletter
\renewenvironment{theindex}
{
    \thispagestyle{fancy}
    \setlength{\parindent}{0pt}
    \let
    \item
    \@idxitem
}{\onecolumn}
\makeatother

\setlength{\parindent}{0cm}

\lstset{
    basicstyle=\ttfamily,
    stringstyle=\ttfamily\color{green!50!black},
    keywordstyle=\bfseries\color{blue},
    commentstyle=\itshape\color{red!50!black},
    showstringspaces=true,
    tabsize=4,
    frame=single,
    numbers=left,
    numberstyle=\tiny,
    firstnumber=1,
    stepnumber=1,
    numbersep=5pt,
    breaklines=true
}

\graphicspath{{img/}}

\pagestyle{fancy}
\setlength{\headheight}{14pt}
\renewcommand{\headrulewidth}{0.5pt}
\lhead{Yannick Brodard}
\chead{}
\rhead{\today}
\renewcommand{\footrulewidth}{0.5pt}
\lfoot{CFPT - École d'informatique}
\cfoot{Cahier des charges}
\rfoot{Page \thepage ~sur \pageref{LastPage}}

% Niveau table des matières
\setcounter{tocdepth}{3}

% ---------------------------------------------------------------------

% Titre
\title{
\Huge{Étude de Unreal Engine 4} \\[0.5cm]
\LARGE{Cahier des charges}
}
\author{Yannick Brodard}
\date{\today}

\begin{document}
\maketitle
\begin{center}
    \includegraphics[scale=0.4]{UE4_logo}
\end{center}
\thispagestyle{empty}
\newpage
\section{Objectifs du projet}
Les objectifs de ce travail sont :
\begin{itemize}
\item Prendre en main l'environnement Unreal Engine 4
\item Mettre en avant les fonctionnalités intéressantes de cet environnement
\item Réaliser un mini-jeu
\item Produire une documentation technique du projet
\item Maintenir un journal de bord pendant la réalisation du projet
\end{itemize}
\section{Description détaillée}
Le but est d'étudier l'environnement de Unreal Engine 4 dans ses détails, c'est-à-dire, la programmation avec UE4, la modélisation, l'édition de différents scénarios et prise en main de la physique du moteur.

Une étude comparative du moteur de jeu sera faite avec le moteur Unity 3D, cette étude est confiée à Monsieur Daniel Lopes. Les critères généraux sont les suivants :\\

\begin{itemize}
\item Programmation
	\begin{itemize}
	\item Langage
	\item Facilité d'intégration
	\end{itemize}
\item Qualité du rendu graphique
\item La création de scène
	\begin{itemize}
	\item Facilité de création
	\end{itemize}
\item Physiques
\item Animations
\item Déploiement
	\begin{itemize}
	\item Facilité de déploiement
	\item Plateformes disponibles
	\item Fonctionnalités\\
	\end{itemize}
\end{itemize}

Le mini-jeu est basé sur le jeu \textit{Galaga}. Dans ce jeu 2D, l'utilisateur contrôle un vaisseau qui peut se déplacer sur les axes x et y. Des vagues de vaisseaux ennemis se positionnent devant le joueur pour tenter de l'éliminer. Ils se positionnent en groupe et tir sur le joueur. Le joueur doit faire de son mieux pour esquiver les tirs et doit détruire tous ses ennemis. 

\begin{figure}[H]
	\begin{center}
	\includegraphics[scale=.9, angle=90]{Galaga}
	\caption{Une capture d'écran du jeu Galaga (Orienter à 90\degre)}
	\end{center}
\end{figure}

Un travail avec la souris \textit{Novint Falcon 3D Touch Controller} est aussi demandé, cette souris 3D permet à l'utilisateur de reproduire des mouvement sur 3 axes qui peuvent être interprétées par l'ordinateur.
Plusieurs possibilités d'intégrations au mini-jeu sont possibles :\\

\begin{itemize}
\item Contrôle de la caméra 3D
\item Contrôle du vaisseau avec des fonctionnalités plus riche qu'une simple souris.
\item etc...
\end{itemize}

\begin{figure}[H]
	\begin{center}
	\includegraphics[scale=.5]{falcon}
	\caption{Souris Novint Falcon 3D Touch Controller}
	\end{center}
\end{figure}
\section{Inventaire du matériel}
\begin{itemize}
\item PC 1x
	\begin{itemize}
	\item 1024MB NVIDIA GeForce GTX 550 Ti
	\item Intel Core i7 2600K @ 3.40GHz
	\item ASUSTeK Computer INC. P8Z68-V LE
	\item 8.00 Go Dual-Channel DDR3 @ 824MHz
	\end{itemize}
\item Écran 2x
\item Clavier 1x
\item Souris 1x
\end{itemize}
\section{Inventaire des logiciels}
\begin{itemize}
\item Unreal Engine 4
\end{itemize}
\section{Éléments mesurables (servant à l'évaluation)}
Réalisation des objectifs, mesurés selon la grille d'évaluation.
\section{Durée}
Le projet se déroulera sur 48 heures de cours, séparée sur 12 jours et 4 périodes par jour. C'est-à-dire, une fois par semaine, le mercredi matin, 4 périodes seront dédiées à ce projet.
\section{Délivrables}
\begin{itemize}
\item Un journal de bord
\item Une copie de la documentation
\item Un poster
\item La documentation et la source du projet seront déposées sur la plateforme Moodle du CFPT-EI.
\end{itemize}
\end{document}
