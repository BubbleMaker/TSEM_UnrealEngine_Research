\documentclass[11pt, a4paper, oneside]{article}

\usepackage[francais]{babel}
\usepackage[T1]{fontenc}
\usepackage[utf8]{inputenc}

\usepackage[left=2cm, right=2cm, top=2cm, bottom=2cm]{geometry}
\usepackage{fancyhdr}
\usepackage{lastpage}
\usepackage{hyperref}

\usepackage{graphicx}
\usepackage{tikz}
\usepackage{pgfgantt}

\usepackage{color}
\usepackage{colortbl}

\usepackage{makeidx}

\usepackage{listings}
\usepackage{CJKutf8}

\makeindex
\makeatletter
\renewenvironment{theindex}
{
    \thispagestyle{fancy}
    \setlength{\parindent}{0pt}
    \let
    \item
    \@idxitem
}{\onecolumn}
\makeatother

\setlength{\parindent}{0cm}

\lstset{
    basicstyle=\ttfamily,
    stringstyle=\ttfamily\color{green!50!black},
    keywordstyle=\bfseries\color{blue},
    commentstyle=\itshape\color{red!50!black},
    showstringspaces=true,
    tabsize=4,
    frame=single,
    numbers=left,
    numberstyle=\tiny,
    firstnumber=1,
    stepnumber=1,
    numbersep=5pt,
    breaklines=true
}

\graphicspath{{img/}}

\pagestyle{fancy}
\setlength{\headheight}{14pt}
\renewcommand{\headrulewidth}{0.5pt}
\lhead{Yannick Brodard}
\chead{}
\rhead{\today}
\renewcommand{\footrulewidth}{0.5pt}
\lfoot{CFPT - École d'informatique}
\cfoot{Étude de Unreal Engine 4}
\rfoot{Page \thepage ~sur \pageref{LastPage}}

% Niveau table des matières
\setcounter{tocdepth}{3}

% ---------------------------------------------------------------------
\begin{document}
\begin{center}
{\Huge{Étude de Unreal Engine 4}} \\[0.5cm]
{\LARGE{Travail de semestre}}\\[0.5cm]
{\Large{Yannick Brodard}}\\[0.3cm]
\today\\
\includegraphics[scale=0.4]{UE4_logo}
\end{center}
Enseignants :
\begin{itemize}
\item Christophe Maréchal
\item Stéphane Garchery
\end{itemize}
\thispagestyle{empty}
\newpage
\tableofcontents
\part{Avant-propos}
\section{Remerciements}
\section{Résumé}
\section{Énoncé}
\section{Introduction}
\subsection{Définition du projet}
\subsection{Définition des objectifs}
\subsection{Préparation au travail de diplôme}
\subsection{Cadre de travail}
\newpage
\part{Analyse préliminaire}
\section{État de l'art}
\subsection{Étude de Unreal Engine 4}
\subsection{Galaga-like}
\newpage
\part{Analyse fonctionnelle}
\section{Unreal Engine 4}
\section{Galaga-like}
\newpage
\part{Analyse Organique}
\section{Galaga-like}
\newpage
\part{Résultats}
\section{Tests}
\section{Conclusion}
\newpage
\part{Annexes}
\end{document}